\documentclass[12pt]{article}

\usepackage[russian]{babel}

\title{Домашняя работа №1}
\author{Артём Левашов}
\date{}

\begin{document}
	\maketitle
	\begin{flushright}
		{\itshape Audi multa,\\loquere pauca}
	\end{flushright}
	
	\vspace{20pt}
	Это мой первый документ в системе компьютерный вёрстки \LaTeX.
	\begin{center}
		\textsf{\huge{<<Ура!!!>>}}
	\end{center}
	\par А теперь формулы. \textsc{Формула}~--- краткое и точное словесное выражение, 
	определение или же ряд математических величин, выраженный условными знаками.
	\vspace{15pt}\\
	\hspace*{28pt}	
	\textbf{\large{Термодинамика}}
	\par Уравнение Менделеева-Клапейрона~--- уравнение состояния идеального газа,
	имеющее вид $pV = \nu RT$, где $p$~--- давление, $V$~--- объем, занимаемый газом, 
	$T$~--- температура газа, $\nu$~--- количество вещества газа, а $R$~--- универсальная 
	газовая постоянная.
	\vspace{15pt}\\
	\hspace*{28pt} \textbf{\large{Геометрия \hfill Планиметрия}}
	\par Для \textsl{плоского} треугольника со сторонами $a,$ $b,$ $c$ и углом $\alpha$, 
	лежащим против стороны $a$, справедливо соотношение
	\[
		a^2 = b^2 + c^2 - 2bc\cos\alpha,
	\]
	из которого можно выразить косинус угла треугольника:
	\[
		\cos\alpha = \frac{b^2 + c^2 - a^2}{2bc}.
	\]
	\par Пусть $p$~--- полупериметр треугольника, тогда путем несложных преобразований
	можно получить, что
	\[
		\tg\frac{\alpha}{2} = \sqrt{\frac{(p-b)(p-c)}{p(p-a)}},	
	\]
	\vspace{1cm}
	\flushleft{На сегодня, пожалуй, хватит\dots Удачи!}
\end{document}
