\documentclass[11pt]{article}

\usepackage[russian]{babel}
\usepackage{amsthm}
\usepackage{amsmath}
\usepackage{amsfonts}
\usepackage{enumerate}
\usepackage[
	a5paper,
	margin = 1cm,
	footskip = 0cm
]{geometry}

\newtheorem{thm}{Теорема}[section]
\newtheorem{definition}{Определение}

\newcounter{num}
\setcounter{num}{1}
\newenvironment{task}[1]
	{{\bfseries \large \underline{Задача \thesection.\arabic{num} #1}}\\}%
	{\stepcounter{num}\\\\}

\DeclareMathOperator{\E}{\mathsf{E}}
\DeclareMathOperator{\D}{\mathsf{D}}
\DeclareMathOperator{\N}{\mathcal{N}}
\DeclareMathOperator{\R}{\mathbb{R}}
\DeclareMathOperator{\Mat}{\text{Mat}}

\begin{document}
	\section{Центральная предельная теорема}
	\begin{thm}[Линдеберга]
		Пусть $\{\xi_k\}_{k \geq 1}$ ~--- независимые случайные величины,
		$\forall k$ $\E\xi_k^2 < +\infty$, обозначим $m_k = \E\xi_k$, $\sigma_k^2 = \D\xi_k > 0$; 
		$S_n = \sum\limits_{i=1}^n \xi_i$; $D_n^2 = \sum\limits_{k=1}^n \sigma_k^2$ и $F_k(x)$~--- 					функция распределения $\xi_k$. Пусть выполнено условие Линдеберга, т.е.
		\[
			\forall\varepsilon > 0\;\frac{1}{\D_n^2}\sum\limits_{k=1}^n\int\limits_{\{x : |x - m_k| > 					\varepsilon D_n\}}(x - m_k)^2 dF_k(x)\underset{n\rightarrow\infty}{\longrightarrow}0.			
		\]  
		Тогда: $\frac{S_n - \E S_n}{\sqrt{\D S_n}}\overset{d}{\rightarrow}\N(0,1), n\rightarrow \infty$	
	\end{thm}
	\section{Гауссовские случайные векторы}
	\begin{definition}
		Случайный вектор $\vec\xi\sim\N(m, \Sigma)$~--- гауссовский, если его характеристическая функция 
		$\varphi_{\vec\xi}(\vec t) = \exp \left(i(\vec m, \vec t) - \frac{1}{2}(\Sigma\vec t, \vec t)
		\right)$, $\vec m \in \R^n$, $\Sigma$~--- симметричная неотрицательно определенная матрица.
	\end{definition}	
	\begin{definition}
		Случайный вектор $\vec\xi$~--- гауссовский, если он представляется в следующем виде: 
		$\vec\xi = A\vec\eta + \vec b$, где $\vec b \in \R^n$, $A \in \Mat (n\times m)$ и $\vec\eta = 
		(\eta_1, \dots, \eta_m)\sim\N(0,1)$ и независимы.
	\end{definition}
	\begin{definition}
		Случайный вектор $\vec\xi$~--- гауссовский, если $\forall\lambda\in\R^n$ случайный вектор
		$(\vec\lambda,\vec\xi)$ имеет нормальное распределение.
	\end{definition}
	\begin{thm}[Об эквивалентости определений гауссовского вектора]
		Предыдущие три определения эквивалентны.
	\end{thm}
	\section{Задачи по астрономии}
	\begin{task}{Венера из Петербурга}
		Параметры орбиты Венеры: большая полуось $ a = 0.7 $ а.е., эксцентриситет $ e = 0 $, наклон к
		плоскости эклиптики $ i = 3^\circ.5 $. Найдите максимально возможную высоту Венеры над 
		горизонтом при наблюдении из Петербурга.
	\end{task}
	\begin{task}{Освещение Марса}
		В далеком будущем для освещения участка поверхности Марса на ареоцентрическую (с центром в центре
		Марса) стационарную орбиту был выведен спутник с массой, равной 1 тонне, на котором был
		установлен постоянно работающий прожектор мощностью 10 МВт, узкий луч которого был направлен
		вниз, на поверхность Марса. Однако оказалось, что для того, чтобы спутник с прожектором совершал
		один оборот ровно за одни марсианские сутки (24 часа 37 минут), радиус его орбиты необходимо
		уменьшить по сравнению с обычным радиусом стационарной орбиты. Насколько потребовалось уменьшить
		радиус орбиты?
	\end{task}
	\begin{task}{Найдите массу}
		Анализ спектра звезды позволил определить ее эффективную температуру $ T $ и ускорение силы
		тяжести на поверхности $ g $. Из наблюдений известны также видимая звездная величина звезды 
		$ m $ и годичный параллакс $ p $ (в угловых секундах). Как, имея эти данные, определить массу
		звезды?
	\end{task}
	\begin{task}{Масса облака}
		Облако в межзвездной среде, состоящее из атомарного водорода, имеет максимальную лучевую
		концентрацию атомов $ 3 \cdot 10^{26} \text{ см}^{-2} $ (количество атомов, находящихся в
		<<столбе>> с основанием $ 1 \text{ см}^2 $). Облако имеет форму шара, плотность газа в облаке
		везде одинакова. При наблюдении облака на длине волны 21 см обнаружилось, что ширина спектральной
		линии составляет $ 0.1 $ мм. Оцените массу облака.
	\end{task}
	\section{Отзыв}
	\begin{enumerate}
		\item[a)] В общих чертах это очень хороший и полезный курс
		\item[b)] Возможно, имеет смысл делать больше примеров для сложных команд (или сразу более 
			сложные примеры)
		\item[c)] Также, наверное, будет полезно делать отдельно шпаргалку со всеми командами, ибо 
			презентации достаточно объемные.
		\item[d)] В плане организации и преподаваемого материала все замечательно
	\end{enumerate}
\end{document}